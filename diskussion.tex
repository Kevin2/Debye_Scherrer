\section{Diskussion}
Die Auswertung des Filmstreifens der  Probe Salz 2 hat ergeben, dass es sich um eine bcc-Struktur
handeln könnte.
Für die Gitterkonstante wurde ein Wert von  $a = \left(4,15\pm0,04\right)\si{\angstrom}$ bestimmt.
Daher könnte es sich um Cäsiumchlorid handeln, da dieses Salz  laut \cite{wiki} eine Gitterkonstante von
$a=4,126\si{\angstrom}$ besitzt.
Die Struktur von Cäsiumchlorid ist allerdings keine bcc-Struktur.
Sie besteht aus zwei unterschiedlich besetzten kubisch primitiven Gittern, die um eine halbe Raumdiagonale gegeneinander versetzt sind.
Die Koordinaten der Atome der Elementarzelle einer Cäsiumchlorid-Struktur stimmen allerdings mit den Koordinaten der Atome einer bcc-Struktur überein.
Damit sind die Ergebnisse der Struktur und der Gitterkonstanten der Probe Salz 2 konsistent. \\
 ~\\
Für die Probe Metall 2 hat die Auswertung des Filmstreifens ergeben, dass es sich um eine fcc-Struktur handelt.
Allerdings konnten nicht alle bei einer fcc-Struktur zu erwartenden Reflexe gemessen werden.
Die Gitterkonstante wurde zu $a = \left(4,73\pm0,02\right)\si{\angstrom}$ bestimmt.
Ein Metall mit der gleichen Gitterstruktur und einer Gitterkonstante von ähnlichem Wert ist Germanium.
Germanium hat eine Gitterkonstante von $a = 5,66\si{\angstrom}$\cite{kiel}.
Da dieser Wert allerdings nicht in der Standardabweichung liegt, ist es sehr fraglich, ob es sich tatsächlich um Germanium handelt.
Allerdings konnte kein anderes Metall mit einer Gitterkonstante von ungefähr  $a = \left(4,73\pm0,02\right)\si{\angstrom}$ gefunden werden.\\
Da es schon bei der Bestimmung der Kristallstruktur Probleme auftraten, ist es möglich, dass  bereits diese falsch bestimmt wurde und daher auch keine korrekte Bestimmung der Gitterkonstanten mehr möglich war.
Eine mögliche Ursache für die Probleme bei der Filmauswerteung ist eine mögliche Verunreinigung der Probe.
