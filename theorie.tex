\section{Theoretische Beschreibung von Kristallstrukturen}
Durch das Debye-Scherrer-Verfahren soll die Kristallstruktur zweier Proben bestimmt werden.
Dazu wird zunächst allgemein auf die theoretische Beschreibung von Kristallstrukturen eingegangen.\\
\\
Eine Kristallstuktur lässt sich durch ein Punktgitter beschreiben.
Ein einzelner Gitterpunkt besteht aus einem einzelnen Atom oder einer Atomgruppe.
Dieses Atom oder diese Atomgruppe wird als Basis bezeichnet.
Das Gitter beschreibt die periodische Anordnung der Atome im Kristall.
Diese Anordnung lässt sich durch die Basisvektoren $\vec a_i$ ($i=1,2,3$) beschreiben.
Durch Linearkombinationen dieser Vektoren lässt sich jeder Gitterpunkt von einer beliebigen Basis aus erreichen.\\
Das von den Vektoren aufgespannte Volumen heißt Elementarzelle.
Befindet sich in dieser Elementarzelle nur ein Gitterpunkt ist die Elementarzelle primitiv.
Betrachtet man nur die Symmetrieeigenschaften der Kristallstrukturen gibt es 14 unterschiedliche Gittertypen, die Bravais-Gitter genannt werden.\\
Im folgenden Unterkapitel \ref{sec:kubisch} wird genauer auf die kubischen Kristallstrukturen eingegangen.
Zudem wird im Unterkapitel \ref{sec:miller} auf die Kennzeichnung von Netzebenen durch Millersche Indizes sowie auf die Berechnung von Abständen zwischen den Netzebenen eingegangen.
Da mithilfe von Beugung von Röntgenstrahlen an Kristallgittern auf die Struktur dieser Kristallgitter rückgeschlossen werden kann, wird im Unterkapitel \ref{sec:Beugung} auf die theoretische Beschreibung dieser Methode eingangen.

\subsection{Kubische Kristallstukturen}
\label{sec:kubisch}
Das kubische Kristallgitter ist das Bravais Gitter, welches in der Natur am häufigsten vorkommt.
Das kubisch-primitive Gitter besteht aus einer Würfelstruktur, wobei sich an jedem Eckpunkt des Würfels ein Gitterpunkt befindet.
Enthält die Gitterzelle  zusätzlich einen Gitterpunkt zentriert in der Mitte des Würfels, heißt die Gitterstruktur kubisch-raumzentriert.
Wenn sich stattdessen neben den Eckatomen jeweils ein zusätzlicher Giterpunkt in der Mitte jede Würfelfläche befindet, wird die Struktur kubisch-flächenzentriert genannt.\\
Einige wichtige Kristalltrukturen, wie beispielsweise die Dimant- oder die Flourid\-/Struktur, setzten sich aus kubisch-flächenzentrierten Strukturen zusammen.
\textcolor{red}{noch was ergänzen?}

\subsection{Gitterebenen}
\label{sec:miller}
Als Gitterebene wird eine Ebene bezeichnet die durch die Gitterpunkte des Kristallstruktur aufgespannt wird.
Aufgrund der Symmetrie des Kristalls gehört zu jeder Netzebene eine Netzebenenscharr.
Alle Netzebenen einer Netzebenenscharr sind parallel zueinander und äquidistant.
Die Lage einer Netzebenenscharr im Raum wird durch die Millerschen Indizes $(hkl)$ festgelegt.
Für die Berechnung der Millerschen Indizes wird das Reziproke der Achsenabschitte der entsprechenden Ebene mit einem beliebigen Faktor ganzzahlig gemacht.
Zwischen zweier benachbarten Netzebenen der gleichen Netzebenenscharr befindet sich der Abstand
\begin{equation}
  d=\frac{a}{\sqrt{h^2+k^2+l^2}},
\end{equation}
wobei $a$ der Betrag der Basisvektoren ist.

\subsection{Beugung von Röntgenstrahlen an Kristallen}
\label{sec:Beugung}
