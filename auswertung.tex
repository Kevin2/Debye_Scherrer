\section{Auswertung}

Zur Auswertung der Messdaten werden die üblichen Formeln für den Mittelwert und die Standardabweichung verwendet. Die Fehlerrechnung wird mittels
Gaußscher Fehlerfortpflanzung durchgeführt. Zusätzlich werden Python 3.5.2 für die Berechnungen, matplotlib.pyplot zum Erstellen der Graphen und
scipy.odr zur Durchführung der lineare Ausgleichsrechnung benutzt.

\subsection{Bestimmung der Gitterstruktur mit Hilfe der Braggreflexe}



Der entwickelte Film enthält kreiförmige Braggreflexe, die charakteristisch für die jeweillige Probe sind. Anhand der Reflexe kann auf die Gitterstruktur der
Probe geschlossen werden und die Gitterkonstante kann bestimmt werden.
Als Referenzwerte werden die Miller'schen Indizes in Gleichung \eqref{eq:streuamplitude} variert und die Reflexe ermittelt, für die die Streuamplitude nicht verschwindet.
Die Ergebnisse für
verschiedene Gitterstrukturen sind in Tabelle \ref{tab:referenz} dargestellt.\newline
Nun sollen die Reflexe auf dem Film mit den Referenzwerten in der Tabelle verglichen werden. Dafür wird der Abstand der Reflexe vom Eintrittsloch der Kamera bestimmt. Ein
Vorteil dabei besteht darin, dass die Kamera einen Umfang von genau $\SI{360}{\milli\meter}$ besitzt. Somit können die Abstände direkt als Winkelmaß
$2\theta$ abgelesen und in Gradmaß umgerechnet werden. Mit Hilfe der Braggbedingung \eqref{eq:bragg} und der Wellenlänge der Röntgenstrahlung von
$\lambda = \SI{1,54286}{\angstrom}$, welche sich aus dem Mittelwert der $K_\alpha$- und der

\begin{table}[H]
  \centering
  \begin{subtable}{.49\textwidth}
      \centering
      \begin{tabular}{
          S[table-format=3.0]
          S[table-format=2.0]
          S[table-format=1.3]}
          \toprule
          $\text{hkl}$ & $\text{N}$ & $\sqrt[]{N_\text{i}\, / \, N_1}$ \\ \midrule
          100 & 1   & 1,000 \\
          110 & 2   & 1,414 \\
          111 & 3   & 1,732 \\
          200 & 4   & 2,000 \\
          210 & 5   & 2,236 \\
          211 & 6   & 2,449 \\
          220 & 8   & 2,828 \\
          221 & 9   & 3,000 \\
          300 & 9   & 3,000 \\
          310 & 10  & 3,162 \\
          311 & 11  & 3,317 \\
          222 & 12  & 3,464 \\
          320 & 13  & 3,606 \\
          321 & 14  & 3,742 \\
          400 & 16  & 4,000 \\
          \bottomrule
      \end{tabular}
      \caption{sc-Gitterstruktur}
    \end{subtable}
    \begin{subtable}{.49\textwidth}
        \centering
    \begin{tabular}{
        S[table-format=3.0]
        S[table-format=2.0]
        S[table-format=1.3]}
        \toprule
        $\text{hkl}$ & $\text{N}$ & $\sqrt[]{N_\text{i}\, / \, N_1}$ \\ \midrule
        110 & 2   & 1,000 \\
        200 & 4   & 1,414 \\
        211 & 6   & 1,732 \\
        220 & 8   & 2,000 \\
        310 & 10  & 2,236 \\
        222 & 12  & 2,449 \\
        321 & 14  & 2,646 \\
        400 & 16  & 2,828 \\
        330 & 18  & 3,000 \\
        411 & 18  & 3,000 \\
        420 & 20  & 3,162 \\
        332 & 22  & 3,317 \\
        422 & 24  & 3,464 \\
        431 & 26  & 3,606 \\
        510 & 26  & 3,606 \\
        \bottomrule
    \end{tabular}
    \caption{bcc-Gitterstruktur}
  \end{subtable}
  \begin{subtable}{.49\textwidth}
      \centering
        \vspace*{5mm}
      \begin{tabular}{
          S[table-format=3.0]
          S[table-format=2.0]
          S[table-format=1.3]}
          \toprule
          $\text{hkl}$ & $\text{N}$ & $\sqrt[]{N_\text{i}\, / \, N_1}$ \\ \midrule
          111 &  3  & 1,000 \\
          200 &  4  & 1,155 \\
          220 &  8  & 1,633 \\
          311 & 11  & 1,915 \\
          222 & 12  & 2,000 \\
          400 & 16  & 2,309 \\
          331 & 19  & 2,517 \\
          420 & 20  & 2,582 \\
          422 & 24  & 2,828 \\
          333 & 27  & 3,000 \\
          511 & 27  & 3,000 \\
          440 & 32  & 3,266 \\
          531 & 35  & 3,416 \\
          442 & 36  & 3,464 \\
          600 & 36  & 3,464 \\
          \bottomrule
      \end{tabular}
      \caption{fcc-Gitterstruktur}
    \end{subtable}
    \begin{subtable}{.49\textwidth}
      \vspace*{5mm}
        \centering
    \begin{tabular}{
        S[table-format=3.0]
        S[table-format=2.0]
        S[table-format=1.3]}
        \toprule
        $\text{hkl}$ & $\text{N}$ & $\sqrt[]{N_\text{i}\, / \, N_1}$ \\ \midrule
        111 & 3  & 1,000 \\
        220 & 8  & 1,633 \\
        311 & 11 & 1,915 \\
        400 & 16 & 2,309 \\
        331 & 19 & 2,517 \\
        422 & 24 & 2,828 \\
        333 & 27 & 3,000 \\
        511 & 27 & 3,000 \\
        440 & 32 & 3,266 \\
        531 & 35 & 3,416 \\
        620 & 40 & 3,651 \\
        533 & 43 & 3,786 \\
        444 & 48 & 4,000 \\
        551 & 51 & 4,123 \\
        711 & 51 & 4,123 \\
        \bottomrule
    \end{tabular}
    \caption{Diamant-Gitterstruktur}
  \end{subtable}
    \caption{Auftretende Beugungsreflexe bei verschiedenen Gittertypen. Es werden jeweils die Miller'schen Indizes $hkl$, die Summe $N$ der einzelnen Werte $h$, $k$, $l$
    und der Quotient $\sqrt[]{N_\text{i}/N_1}$
    angegeben. Die Tabellen dienen als Referenzwerte, um die Kristallstruktur der Probe zu ermitteln.}
    \label{tab:referenz}
\end{table}


$K_\beta$-Linie ergibt, können die Netzebenenabstände bestimmt werden. Anschließen wird das Verhältnis $d_1 / d_\text{i}$ für die verschiedenen Reflexe gebildet.\newline
Damit die Gitterkonstante der Probe bestimmt werden kann, wird Gleichung \eqref{eq:d} in Gleichung \eqref{eq:bragg} eingesetzt. Es ergibt sich der Ausdruck
\begin{equation}
  \frac{\lambda}{2\sin(\theta)}\,\,\sqrt[]{h^2+k^2+l^2}=d\,\,\sqrt[]{N}=a.
\end{equation}
Durch Normieren dieses Ausdruck kann der Zusammenhang
\begin{equation}
\frac{d_1}{d_\text{i}}=\sqrt[]{N_\text{i}\, / \, N_1}
\end{equation}
ermittelt werden. Es wird deutlich, dass die gebildeten Verhältnisse $d_1 / d_\text{i}$ mit den Ausdrücken $\sqrt[]{N_\text{i}\, / \, N_1}$ aus Tabelle
\ref{tab:referenz} verglichen werden können. Somit kann die Gitterstruktur der Probe ermittelt werden.
Mit bekannter Gitterstruktur kann anschließend die Gitterkonstante der Probe berechnet werden. Es werden aus den einzelnen Messwerten die verschiedenen
Gitterkonstanten ermittelt und anschließend gegen $\cos^2(\theta)$ aufgetragen. Dann wird eine lineare Ausgleichsrechnung durchgeführt, der y-Achsenabschnitt
liefert dabei den Wert für die Gitterkonstante. Bei der linearen Ausgleichsrechnungen werden sowohl die Unsicherheiten auf die x-Werte, als auch die Unsicherheiten auf
den y-Werten berücksichtigt.

\subsection{Ergebnisse für Metall 2}



\begin{figure}[H]
  \centering
  \includegraphics[scale=0.5]{probe1.pdf}
  \caption{metall}
  \label{abb:metall}
\end{figure}

\subsection{Ergebnisse für Salz 2}

In Tabelle \ref{tab:salz} sind die Messwerte für die Probe Salz 2 zu sehen. Zusätzlich sind die berechneten Winkel, die berechneten Gitterabstände und die
daraus resultierenden Verhältnisse $d_1 / d_\text{i}$ zu erkennen.
\begin{table}
  \centering
\begin{tabular}{
  S[table-format=2.1(1)]
  S[table-format=2.2(2)]
  S[table-format=1.3(3)]
  S[table-format=1.2(2)]}
  \toprule
  $\text{r / mm}$ & $\theta\text{ / } ^\circ$ &$\text{d}$ &  $\text{bla}$  \\ \midrule

  \bottomrule
\end{tabular}
\caption{Messwerte und Ergebnisse für die Probe Salz 2. Es sind die Abstände der Braggreflexe, die Winkel $\theta$, die Netzebenenabstände, sowie die Verhältnisse
$d_1 / d_\text{i}$ zu erkennen. In der letzten Spalte befinden sich die jeweiligen Werte für die Gitterkonstante $a$.}
\label{tab:salz}
\end{table}




Ein Vergleich von $d_1 / d_\text{i}$ mit den $\sqrt[]{N_\text{i}\, / \, N_1}$-Spalten in Tabelle
\ref{tab:referenz} zeigt, dass es sich bei der Probe um eine bcc-Gitterstruktur handelt. Es könnte sich somit um eine Cäsiumchlorid-Struktur handeln. Somit werden die
Referenzwerte für die bcc-Gitterstruktur zur Berechnung der Gitterkonstanten verwendet. Die Werte für die Gitterkonstanten sind in der letzten Spalte von
Tabelle \ref{tab:salz} zu finden. In Abbildung \ref{abb:salz} sind die einzelnen Gitterkonstanten gegen $\cos^2(\theta)$ aufgetragen. Die lineare Ausgleichsrechnung liefert
einen y-Achsenabschnitt von
\begin{equation}
  a = 4,02\pm0,03.
\end{equation}
Dieser Wert ist das Ergebnis für die Gitterkonstante der Probe.

\begin{figure}[H]
  \centering
  \includegraphics[scale=0.5]{probe2.pdf}
  \caption{salz}
  \label{abb:salz}
\end{figure}
