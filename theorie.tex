\section{Theoretische Beschreibung von Kristallstrukturen}
Durch das Debye-Scherrer-Verfahren soll die Kristallstruktur zweier Proben bestimmt werden.
Dazu wird zunächst allgemein auf die theoretische Beschreibung von Kristallstrukturen eingegangen.\\
\\
Eine Kristallstuktur lässt sich durch ein Punktgitter beschreiben.
Ein einzelner Gitterpunkt besteht aus einem einzelnen Atom oder einer Atomgruppe.
Dieses Atom oder diese Atomgruppe wird als Basis bezeichnet.
Das Gitter beschreibt die periodische Anordnung der Atome im Kristall.
Diese Anordnung lässt sich durch die Basisvektoren $\vec a_i$ ($i=1,2,3$) beschreiben.
Durch Linearkombinationen dieser Vektoren lässt sich jeder Gitterpunkt von einer beliebigen Basis aus erreichen.\\
Das von den Vektoren aufgespannte Volumen heißt Elementarzelle.
Befindet sich in dieser Elementarzelle nur ein Gitterpunkt ist die Elementarzelle primitiv.
