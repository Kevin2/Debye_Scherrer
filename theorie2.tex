\subsection{Debye-Scherrer-Verfahren}

Wird eine kristalline Probe mit monochromatischem Röntgenlicht bestrahlt, kann anhand des Beugungsmusters eine Aussage über die Gitterstruktur getroffen werden.
Die Röntgenstrahlung wird an den Verschiedenen Netzebenen um den Beugungsinkel $\theta$ gestreut. Bei unterschiedlichen Gitterstrukturen löschen sich die
Streuamplituden bestimmter Netzebenen aus. Dadurch kann Aufschluss über die Kristallstruktur gewonnen werden. Da Reflexionen nur auftreten, wenn der Bragg-Winkel getroffen wird,
wird die zu unteruschende Probe zerkleinert. Dies ermöglicht eine statistische Verteilung der Ausrichtungen der Kristalle, wodurch die Bragg-Reflexionen ermöglicht werden.\newline
Beim Debye-Scherrer-Verfahren können unterschiedliche systematische Fehler auftreten. Da die Beugung nur an einem schmalen Streifen der Probe stattfindet, werden die Winkel zu groß gemessen.
Dieser Fehler ist vor allem bei kleinen Winkel ausschlaggebend. Um die Gitterkonstante richtig bestimmen zu können, wird der Korrekturfaktor
\begin{equation}
  \frac{\Delta a_\text{A}}{a} = \frac{\rho}{2R}\left(1 - \frac{R}{F} \right) \frac{\cos^2(\theta)}{\theta}
\end{equation}
eingeführt. Dabei ist $\rho$ der Probenradius, $R$ der Kameraradius und $F$ der Abstand vom Fokus zur Probe. Bei großen Winkel tritt ein weiterer systematischer Fehler auf. Die Probenachse liegt
nicht genau auf der Achse des Films was zur Verschiebung der Strahlen führt. Aus diesem Grund wird eine weitere Korrektur, welche explizit
\begin{equation}
  \frac{\Delta a_\text{V}}{a} = \frac{V}{R} \cos^2(\theta)
\end{equation}
dargestellt werden kann, zur Bestimmung der Gitterkonstante verwendet. Um beide Korrekturen zu berücksichtigen wird der Ausdruck
\begin{equation}
  \Delta a_\text{ges} = \Delta a_\text{V} + \Delta a_\text{A}
\end{equation}
gebildet, welcher proportional zu $\cos^2$ ist. Durch das Auftragen der berechneten Gitterkonstanten gegen $\cos^2$ kann mit Hilfer einer linearen Ausgleichsrechnung die beste
Gitterkonstante bestimmt werden, da sich ein linearer Zusammenhang zwischen $a(\theta)$ und $\cos^2(\theta)$ zeigt.

\subsection{Charackeristika der Röntgenröhre}

In der Röntgenröhre entsteht sowohl $K_\alpha$-Strahlung, als auch $K_\beta$-Strahlung. Da die $K_\beta$-Strahlung für den Versuch nicht verwendet werden soll, wird diese mit Hilfe eines Filters entfernt.
Die Anwendung des Filters führt dazu, dass die $K_\alpha$-Strahlung in die $K_{\alpha_1}$-Linie und in die $K_{\alpha_2}$-Linie unterteilt wird, welche sich jedoch nicht mehr von einander trennen
lassen. Dies führt bei hohen Reflexionswinkeln zu einer Ringaufspaltung, die bei der Auswertung des Beugungsmusters berücksichtigt werden muss.
